\documentclass[11pt,dvips]{article}
%\documentclass[12pt,dvips]{article}
%\documentclass[preprint,dvips,numbers,sort&compress]{elsarticle}
%\documentclass[12pt,review,dvips,numbers,sort&compress]{elsarticle}
% Geometry.
\usepackage{geometry}
\geometry{a4paper,
left=1.5cm,
right=1.5cm,
top=1.5cm,
bottom=1.5cm,
}
% Global functionalities.
\usepackage[numbers,sort&compress]{natbib}
\usepackage{hyperref}
% encoding.
\usepackage[utf8]{inputenc}
% Mathematics.
\usepackage{amsmath}
\usepackage{amssymb}
\usepackage{amsfonts}
\usepackage{amsthm}
\usepackage{arydshln}
% Authoring.
%\usepackage{authblk}
\usepackage{graphicx}
\usepackage{subfigure}
\usepackage{color}
\usepackage{paralist}
\usepackage{multirow}
\usepackage{booktabs}
\usepackage{setspace}
%\usepackage{lmodern}

%\doublespacing

\graphicspath{{thingol_eps/}}

\renewcommand{\figurename}{Fig.}
\newcommand{\topcaption}{%
\setlength{\abovecaptionskip}{0pt}%
\setlength{\belowcaptionskip}{10pt}%
\caption}
\numberwithin{equation}{section}

\newcommand{\defeq}{\ensuremath{\buildrel {\text{def}}\over{=}}} 

\title{
%
The Conservation Element and Solution Element Method and Unstructured Meshes
%
}

\author{
%
Yung-Yu Chen
%
}

\begin{document}

\maketitle

\begin{abstract}
%
Describe how the unstructured meshes work for the conservation element and
solution element (CESE) method and SOLVCON.
%
\end{abstract}

\section{Unstructured Meshes}
%
\label{s:ust_intro}

The conservation element and solution element (CESE) method is developed
against the set-up of unstructured meshes in multi-dimensional
space\cite{mavriplis_unstructured_1997, wang_2d_1999}.  Comparing to structured
meshes, unstructured meshes enable flexible connectivity and solvers can thus
use simplices.  A practical implementation further allows mixing elements of
different shapes.  Different regions may choose the suitable shape of elements.

The implementation of unstructured meshes dictate simulation software operates.
Code for unstructured meshes serves two use cases: numerical methods for
simulation and mesh generation.  In simulation software the former is usually
emphasized.  In the time being, SOLVCON is the case, although the latter is
being noticed.

Unstructured-mesh code should be organized in a library for ease of
utilization.  It should provide three functionalities:
%
\begin{enumerate}
%
\item Resource management: allocation, deallocation, and tracking.
%
\item Geometrical entity creation, read, update, and deletion (CRUD).
%
\item Spatial indexing.
%
\end{enumerate}

\clearpage
%
\section{Meta Data}

The basic meta data for the elements are specified in Table~\ref{t:elm:meta}.

%%%%%%%%%%%%%%%%%%%%%%%%%%%%%%%%%%%%%%%%%%%%%%%%%%%%%%%%%%%%%%%%%%%%%%%%%%%%%%%%
%
\begin{table}[h]
\centering

\topcaption{
%
Meta data of the unstructured-mesh elements.
%
}

\label{t:elm:meta}
\begin{tabular}{lrrrrr}
\toprule
Name          & Type & Dimension & Number of & Number of & Number of \\
              &      &           &    Points &     Lines &  Surfaces \\
\midrule
Point         & 0    & 0         & 1         & (n/a) 0   & (n/a) 0   \\
Line          & 1    & 1         & 2         & (n/a) 0   & (n/a) 0   \\
Quadrilateral & 2    & 2         & 4         & 4         & (n/a) 0   \\
Triangle      & 3    & 2         & 3         & 3         & (n/a) 0   \\
Hexahedron    & 4    & 3         & 8         & 12        & 6         \\
Tetrahedron   & 5    & 3         & 4         & 6         & 4         \\
Prism         & 6    & 3         & 6         & 9         & 5         \\
Pyramid       & 7    & 3         & 5         & 8         & 5         \\
\bottomrule
\end{tabular}
\end{table}
%
%%%%%%%%%%%%%%%%%%%%%%%%%%%%%%%%%%%%%%%%%%%%%%%%%%%%%%%%%%%%%%%%%%%%%%%%%%%%%%%%

%%%%%%%%%%%%%%%%%%%%%%%%%%%%%%%%%%%%%%%%%%%%%%%%%%%%%%%%%%%%%%%%%%%%%%%%%%%%%%%%
%
\begin{figure}
\centering
\subfigure[]{
  \includegraphics{elm_ln.eps}
  \label{f:elm2d:ln}
}
\subfigure[]{
  \includegraphics{elm_quad.eps}
  \label{f:elm2d:quad}
}
\subfigure[]{
  \includegraphics{elm_tri.eps}
  \label{f:elm2d:tri}
}

\caption{
%
Node definition of one- and two-dimensional mesh elements in SOLVCON:
%
\subref{f:elm2d:ln} Line (type number 1).
%
\subref{f:elm2d:quad} Quadrilateral (type number 2).
%
\subref{f:elm2d:tri} Triangle (type number 3).
%
}

\label{f:elm2d}
\end{figure}
%
%%%%%%%%%%%%%%%%%%%%%%%%%%%%%%%%%%%%%%%%%%%%%%%%%%%%%%%%%%%%%%%%%%%%%%%%%%%%%%%%

%%%%%%%%%%%%%%%%%%%%%%%%%%%%%%%%%%%%%%%%%%%%%%%%%%%%%%%%%%%%%%%%%%%%%%%%%%%%%%%%
%
\begin{figure}
\centering
\subfigure[]{
  \includegraphics{elm_hex.eps}
  \label{f:elm3d:hex}
}
\subfigure[]{
  \includegraphics{elm_tet.eps}
  \label{f:elm3d:tet}
}
\subfigure[]{
  \includegraphics{elm_psm.eps}
  \label{f:elm3d:psm}
}
\subfigure[]{
  \includegraphics{elm_pym.eps}
  \label{f:elm3d:pym}
}

\caption{
%
Node definitions of three-dimensional mesh elements in SOLVCON:
%
\subref{f:elm3d:hex} Hexahedron (type number 4).
%
\subref{f:elm3d:tet} Tetrahedron (type number 5).
%
\subref{f:elm3d:psm} Prism (type number 6).
%
\subref{f:elm3d:pym} Pyramid (type number 7).
%
}

\label{f:elm3d}
\end{figure}
%
%%%%%%%%%%%%%%%%%%%%%%%%%%%%%%%%%%%%%%%%%%%%%%%%%%%%%%%%%%%%%%%%%%%%%%%%%%%%%%%%

%%%%%%%%%%%%%%%%%%%%%%%%%%%%%%%%%%%%%%%%%%%%%%%%%%%%%%%%%%%%%%%%%%%%%%%%%%%%%%%%
%
\begin{table}
\centering

\topcaption{
%
The relation between a one-dimensional element and its sub-entities.
%
}

\label{t:subent1d}
\begin{tabular}{lll}
\toprule
Shape (type) & Face & = Node \\
\midrule
Line (1)     & 0    & 0      \\
             & 1    & 1      \\
\bottomrule
\end{tabular}
\end{table}
%
%%%%%%%%%%%%%%%%%%%%%%%%%%%%%%%%%%%%%%%%%%%%%%%%%%%%%%%%%%%%%%%%%%%%%%%%%%%%%%%%

%%%%%%%%%%%%%%%%%%%%%%%%%%%%%%%%%%%%%%%%%%%%%%%%%%%%%%%%%%%%%%%%%%%%%%%%%%%%%%%%
%
\begin{table}
\centering

\topcaption{
%
The relations between two-dimensional elements and their sub-entities.  Both of
two-dimensional elements are enclosed by straight lines.  Node orientation of
two-dimensional elements is defined in Fig.~\ref{f:elm2d}.
%
}

\label{t:subent2d}
\begin{tabular}{lll}
\toprule
Shape (type)      & Face & = Line formed by nodes \\
\midrule
Quadrilateral (2) & 0    & $\diagup$ 0 1          \\
                  & 1    & $\diagup$ 1 2          \\
                  & 2    & $\diagup$ 2 3          \\
                  & 3    & $\diagup$ 3 0          \\
\midrule
Triangles (3)     & 0    & $\diagup$ 0 1          \\
                  & 1    & $\diagup$ 1 2          \\
                  & 2    & $\diagup$ 2 0          \\
\bottomrule
\end{tabular}
\end{table}
%
%%%%%%%%%%%%%%%%%%%%%%%%%%%%%%%%%%%%%%%%%%%%%%%%%%%%%%%%%%%%%%%%%%%%%%%%%%%%%%%%

%%%%%%%%%%%%%%%%%%%%%%%%%%%%%%%%%%%%%%%%%%%%%%%%%%%%%%%%%%%%%%%%%%%%%%%%%%%%%%%%
%
\begin{table}
\centering

\topcaption{
%
The relation between a one-dimensional element and its sub-entities.
%
}

\label{t:subent1d}
\begin{tabular}{lll}
\toprule
Shape (type) & Face & = Node \\
\midrule
Line (1)     & 0    & 0      \\
             & 1    & 1      \\
\bottomrule
\end{tabular}
\end{table}
%
%%%%%%%%%%%%%%%%%%%%%%%%%%%%%%%%%%%%%%%%%%%%%%%%%%%%%%%%%%%%%%%%%%%%%%%%%%%%%%%%

%%%%%%%%%%%%%%%%%%%%%%%%%%%%%%%%%%%%%%%%%%%%%%%%%%%%%%%%%%%%%%%%%%%%%%%%%%%%%%%%
%
\begin{table}
\centering

\topcaption{
%
The relations between two-dimensional elements and their sub-entities.  Both of
two-dimensional elements are enclosed by straight lines.  Node orientation of
two-dimensional elements is defined in Fig.~\ref{f:elm2d}.
%
}

\label{t:subent2d}
\begin{tabular}{lll}
\toprule
Shape (type)      & Face & = Line formed by nodes \\
\midrule
Quadrilateral (2) & 0    & $\diagup$ 0 1          \\
                  & 1    & $\diagup$ 1 2          \\
                  & 2    & $\diagup$ 2 3          \\
                  & 3    & $\diagup$ 3 0          \\
\midrule
Triangles (3)     & 0    & $\diagup$ 0 1          \\
                  & 1    & $\diagup$ 1 2          \\
                  & 2    & $\diagup$ 2 0          \\
\bottomrule
\end{tabular}
\end{table}
%
%%%%%%%%%%%%%%%%%%%%%%%%%%%%%%%%%%%%%%%%%%%%%%%%%%%%%%%%%%%%%%%%%%%%%%%%%%%%%%%%

%%%%%%%%%%%%%%%%%%%%%%%%%%%%%%%%%%%%%%%%%%%%%%%%%%%%%%%%%%%%%%%%%%%%%%%%%%%%%%%%
%
\begin{table}
\centering

\topcaption{
%
The relations between three-dimensional elements and their sub-entities.
Three-dimensional elements are enclosed by triangles or quadrilaterals, or a
combination of them.  $\square$ in the third column denotes quadrilaterals,
while $\triangle$ triangles.  Nodes in the third column are ordered so that the
normal vector of a surface points outward from the volume by following
right-hand rule.  Node orientation of three-dimensional elements is defined in
Fig.~\ref{f:elm3d}.
%
}

\label{t:subent3d}
\begin{tabular}{lll}
\toprule
Shape (type)    & Face & = Surface formed by nodes \\
\midrule
Hexahedron (4)  & 0    & $\square$ 0 3 2 1         \\
                & 1    & $\square$ 1 2 6 5         \\
                & 2    & $\square$ 4 5 6 7         \\
                & 3    & $\square$ 0 4 7 3         \\
                & 4    & $\square$ 0 1 5 4         \\
                & 5    & $\square$ 2 3 7 6         \\
\midrule
Tetrahedron (5) & 0    & $\triangle$ 0 2 1         \\
                & 1    & $\triangle$ 0 1 3         \\
                & 2    & $\triangle$ 0 3 2         \\
                & 3    & $\triangle$ 1 2 3         \\
\midrule
Prism (6)       & 0    & $\triangle$ 0 1 2         \\
                & 1    & $\triangle$ 3 5 4         \\
                & 2    & $\square$ 0 3 4 1         \\
                & 3    & $\square$ 0 2 5 3         \\
                & 4    & $\square$ 1 4 5 2         \\
\midrule
Pyramid (7)     & 0    & $\triangle$ 0 4 3         \\
                & 1    & $\triangle$ 1 4 0         \\
                & 2    & $\triangle$ 2 4 1         \\
                & 3    & $\triangle$ 3 4 2         \\
                & 4    & $\square$ 0 3 2 1         \\
\bottomrule
\end{tabular}
\end{table}
%
%%%%%%%%%%%%%%%%%%%%%%%%%%%%%%%%%%%%%%%%%%%%%%%%%%%%%%%%%%%%%%%%%%%%%%%%%%%%%%%%

\clearpage
\section{Two-Dimensional Triangular Mesh}
\label{s:mesh_2d_tri}

Figure~\ref{f:mesh_2d_tri} exhibits 6 triangular mesh elements.  The CESE
method lets the solutions to be evaluated at their geometrical centers and
constructs the solution elements (SEs).  The element centers and the mesh
vertices consist of the conservation elements.  See Fig.~\ref{f:mesh_2d_ce}.

%%%%%%%%%%%%%%%%%%%%%%%%%%%%%%%%%%%%%%%%%%%%%%%%%%%%%%%%%%%%%%%%%%%%%%%%%%%%%%%%
%
\begin{figure}[h]
\centering
\includegraphics{mesh_2d_tri.eps}
\caption{Triangular mesh in two-dimensional space.}
\label{f:mesh_2d_tri}
\end{figure}

\begin{figure}[h]
\centering
\includegraphics{mesh_2d_ce.eps}
\caption{Conservation elements of triangular meshes.}
\label{f:mesh_2d_ce}
\end{figure}
%
%%%%%%%%%%%%%%%%%%%%%%%%%%%%%%%%%%%%%%%%%%%%%%%%%%%%%%%%%%%%%%%%%%%%%%%%%%%%%%%%

\clearpage
\addcontentsline{toc}{chapter}{Bibliography}
\bibliographystyle{plain}
\bibliography{thingol_main}

\end{document}

% vim: set ai et sw=2 ts=2 tw=79:
